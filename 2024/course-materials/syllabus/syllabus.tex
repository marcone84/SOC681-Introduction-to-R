% Options for packages loaded elsewhere
\PassOptionsToPackage{unicode}{hyperref}
\PassOptionsToPackage{hyphens}{url}
%
\documentclass[
  10pt,
]{article}
\usepackage{amsmath,amssymb}
\usepackage{iftex}
\ifPDFTeX
  \usepackage[T1]{fontenc}
  \usepackage[utf8]{inputenc}
  \usepackage{textcomp} % provide euro and other symbols
\else % if luatex or xetex
  \usepackage{unicode-math} % this also loads fontspec
  \defaultfontfeatures{Scale=MatchLowercase}
  \defaultfontfeatures[\rmfamily]{Ligatures=TeX,Scale=1}
\fi
\usepackage[]{mathpazo}
\ifPDFTeX\else
  % xetex/luatex font selection
\fi
% Use upquote if available, for straight quotes in verbatim environments
\IfFileExists{upquote.sty}{\usepackage{upquote}}{}
\IfFileExists{microtype.sty}{% use microtype if available
  \usepackage[]{microtype}
  \UseMicrotypeSet[protrusion]{basicmath} % disable protrusion for tt fonts
}{}
\makeatletter
\@ifundefined{KOMAClassName}{% if non-KOMA class
  \IfFileExists{parskip.sty}{%
    \usepackage{parskip}
  }{% else
    \setlength{\parindent}{0pt}
    \setlength{\parskip}{6pt plus 2pt minus 1pt}}
}{% if KOMA class
  \KOMAoptions{parskip=half}}
\makeatother
\usepackage{xcolor}
\usepackage[margin=1in]{geometry}
\usepackage{graphicx}
\makeatletter
\def\maxwidth{\ifdim\Gin@nat@width>\linewidth\linewidth\else\Gin@nat@width\fi}
\def\maxheight{\ifdim\Gin@nat@height>\textheight\textheight\else\Gin@nat@height\fi}
\makeatother
% Scale images if necessary, so that they will not overflow the page
% margins by default, and it is still possible to overwrite the defaults
% using explicit options in \includegraphics[width, height, ...]{}
\setkeys{Gin}{width=\maxwidth,height=\maxheight,keepaspectratio}
% Set default figure placement to htbp
\makeatletter
\def\fps@figure{htbp}
\makeatother
\setlength{\emergencystretch}{3em} % prevent overfull lines
\providecommand{\tightlist}{%
  \setlength{\itemsep}{0pt}\setlength{\parskip}{0pt}}
\setcounter{secnumdepth}{-\maxdimen} % remove section numbering
\linespread{1.05}
\ifLuaTeX
  \usepackage{selnolig}  % disable illegal ligatures
\fi
\IfFileExists{bookmark.sty}{\usepackage{bookmark}}{\usepackage{hyperref}}
\IfFileExists{xurl.sty}{\usepackage{xurl}}{} % add URL line breaks if available
\urlstyle{same}
\hypersetup{
  pdfauthor={Dr.~Marcus Mann},
  hidelinks,
  pdfcreator={LaTeX via pandoc}}

\title{SOC 681\\
COMPUTATIONAL SOCIAL SCIENCE USING R\\
Purdue University\\
\strut \\
\hspace*{0.333em}Syllabus}
\author{Dr.~Marcus Mann}
\date{Fall 2024}

\begin{document}
\maketitle

\hypertarget{contact-and-logistics}{%
\section{CONTACT AND LOGISTICS}\label{contact-and-logistics}}

E-mail: \texttt{mannml@purdue.edu}

Website: \texttt{https://github.com/marcone84/computational-sociology-1}

Class meetings: Wednesday 1:30-4:20 p.m.

Office hours: by appointment.

\hypertarget{course-description}{%
\section{COURSE DESCRIPTION}\label{course-description}}

This (crash) course is designed to provide a friendly and broad
introduction to computational methods in social science using R. First,
we will discuss the history and evolving role of computational methods
in social science and their attendant implications for conducting
ethical and generative research. Then we will discuss and practice the
most basic programming skills in R including setting working
directories, searching for and loading relevant packages, and basic date
cleaning and manipulation. After this, the course will focus on
introducing and applying a specific method of data collection or
analysis every week. These will include 1) data wrangling, 2) working
with API's, 3) experiments and surveys, 4) text analysis methods, 5) and
data visualization. While this course does aim to broaden your awareness
and skills in various kinds of data collection and analysis, it is not a
statistics course, so do not expect detailed coverage of specific
statistical methods or strategies. In general, we will study and discuss
how computational methods have changed and broadened social scientific
research, when they should and should not be applied, and - most
importantly - how they can help you with your specific research agenda.

\hypertarget{prerequisites-and-preparation}{%
\section{PREREQUISITES AND
PREPARATION}\label{prerequisites-and-preparation}}

This course assumes no experience in R or other programming languages
and is meant to be a gentle introduction to using R for social
scientific research. That said, online resources are readily available
to attain at least a rudimentary understanding of the basic skills
needed to begin coding in R. One such resource is the Summer Institute
in Computational Social Science websit here:
\url{https://sicss.io/overview}

\hypertarget{assessment}{%
\section{ASSESSMENT}\label{assessment}}

100\% of your grade will be based on a ``final paper'' (at least 5,000
words) that uses some kind of computational method covered in this
course. This paper will be assessed by 1) relevance to a particular
field, 2) justification of hypotheses, 3) justification of computational
method, and 3) clear communication of results (including at least one
data viz) and implications. This paper has no other parameters.

\hypertarget{readings}{%
\section{READINGS}\label{readings}}

There are weekly reading assignments for this course. These readings
include methodological texts, reviews of relevant methodological and
theoretical considerations, and examples of how sociologists and other
social scientists apply computational approaches in their research.
Given the range and complexity of some of the approaches we will cover
in the course, I have included a diverse set of readings for each topic.
Some students may find the instructional readings more useful whereas
others may benefit from the academic articles applying these methods to
substantive research questions.

\hypertarget{require-texts-and-useful-references}{%
\subsection{Require texts and useful
references}\label{require-texts-and-useful-references}}

\emph{* indicates a required text. All required texts and useful
references are available for free online on the listed websites.}

\begin{itemize}
\tightlist
\item
  *Matthew Salganik. 2017. \emph{Bit by Bit}. Princeton University
  Press. \url{https://www.bitbybitbook.com/en/1st-ed/preface/}
\item
  *Wickham, Hadley, and Garrett Grolemund. 2016. \emph{R for Data
  Science: Import, Tidy, Transform, Visualize, and Model Data}.
  (\emph{R4DS}). O'Reilly Media, Inc. \url{https://r4ds.had.co.nz/}
\item
  *Silge, Julia, and David Robinson. 2017. \emph{Text Mining with R: A
  Tidy Approach.} O'Reilly Media.
  \url{https://www.tidytextmining.com/dtm.html}.
\item
  *Healy, Kieran. 2018. \emph{Data Visualization: A Practical
  Introduction}. Princeton University Press. \url{https://socviz.co/}
\end{itemize}

\hypertarget{course-policies}{%
\section{COURSE POLICIES}\label{course-policies}}

The Purdue Sociology Department strives to create an environment that
supports and affirms diversity in all manifestations, including race,
ethnicity, gender, sexual orientation, religion, age, social class,
disability status, region/country of origin, and political orientation.
This class will be a space for tolerance, respect, and mutual dialogue.
Students must abide by the Code of Student Conduct at all times,
including during lectures and in participation online.

All students must abide by the university's Academic Integrity Policy.
Violations of academic integrity will result in disciplinary action.

In accordance with University policy, if you have a documented
disability and require accommodations to obtain equal access in this
course, please contact me during the first week of classes. Students
with disabilities must be registered with the Office of Student
Disability Services and must provide verification of their eligibility
for such accommodations.

\hypertarget{course-outline}{%
\section{COURSE OUTLINE}\label{course-outline}}

\textbf{\emph{This outline is tentative and subject to change.}}

\hypertarget{week-1---for-828}{%
\subsection{Week 1 - For 8/28}\label{week-1---for-828}}

\hypertarget{introduction-to-computational-sociology}{%
\subsubsection{Introduction to Computational
Sociology}\label{introduction-to-computational-sociology}}

\emph{Readings}

\begin{itemize}
\tightlist
\item
  \emph{R4DS}: Preface, ``Chapters'' 4-7 on website
\item
  \emph{Bit by Bit}, Chapter 1
\item
  Lazer, David, et al.~2009. ``Computational Social Science.''
  \emph{Science} 323 (5915): 721--23.
  \url{https://www.science.org/doi/10.1126/science.1167742}.
\item
  Edelmann, Achim, Tom Wolff, Danielle Montagne, and Christopher A.
  Bail. 2020. ``Computational Social Science and Sociology.''
  \emph{Annual Review of Sociology} 46 (1):
  \url{https://www.annualreviews.org/doi/abs/10.1146/annurev-soc-121919-054621}
\end{itemize}

\hypertarget{week-2---for-94}{%
\subsection{Week 2 - For 9/4}\label{week-2---for-94}}

\hypertarget{data-structures-wrangling-and-the-tidyverse}{%
\subsubsection{Data Structures, Wrangling, and the
Tidyverse}\label{data-structures-wrangling-and-the-tidyverse}}

\emph{Readings}

\begin{itemize}
\tightlist
\item
  \emph{R4DS}: ``Chapters'' 9-13 on website
\item
  Golder, Scott A., and Michael W. Macy. 2014. ``Digital Footprints:
  Opportunities and Challenges for Online Social Research.''
  \emph{Annual Review of Sociology} 40 (1): 129--52.
  \url{https://doi.org/10.1146/annurev-soc-071913-043145}.
\item
  Bail, Christopher A. 2014. ``The Cultural Environment: Measuring
  Culture with Big Data.'' \emph{Theory and Society} 43 (3--4): 465--82.
  \url{https://doi.org/10.1007/s11186-014-9216-5}.
\end{itemize}

\hypertarget{week-3---for-911}{%
\subsection{Week 3 - For 9/11}\label{week-3---for-911}}

\hypertarget{data-collection-i-apis}{%
\subsubsection{Data Collection I: APIs}\label{data-collection-i-apis}}

\emph{Readings}

\begin{itemize}
\tightlist
\item
  \emph{R4DS}: C11 (``Strings with stringr''), 13 (``Dates and Times
  with lubridate'')
\item
  \emph{Bit by Bit}, C2
\item
  Baumgartner, Jason, Savvas Zannettou, Brian Keegan, Megan Squire, and
  Jeremy Blackburn. 2020. ``The Pushshift Reddit Dataset.'' In
  \emph{Proceedings of the International AAAI Conference on Web and
  Social Media}, 14:830--39.
\item
  Freelon, Deen. 2018. ``Computational Research in the Post-API Age.''
  \emph{Political Communication} 35 (4): 665--68.
  \url{https://doi.org/10.1080/10584609.2018.1477506}.
\end{itemize}

\emph{Recommended}

\emph{I have included recommended readings that use a range of different
APIs including Spotify (Askin and Mauskampf), Facebook (Davidson and
Berezin; Bail, Brown and Mann), Google Trends (Davidson and Berezin;
Bail, Brown, and Wimmer, Gross and Mann), Twitter (Mitts), and YouTube
(Munger and Phillips).}

\begin{itemize}
\tightlist
\item
  Askin, Noah, and Michael Mauskapf. 2017. ``What Makes Popular Culture
  Popular? Product Features and Optimal Differentiation in Music.''
  \emph{American Sociological Review} 82 (5): 910--44.
  \url{https://doi.org/10.1177/0003122417728662}.
\item
  Davidson, Thomas, and Mabel Berezin. 2018. ``Britain First and the UK
  Independence Party: Social Media and Movement-Party Dynamics.''
  \emph{Mobilization: An International Quarterly} 23 (4): 485--510.
  \url{https://doi.org/10.17813/1086-671X-23-4-485}.
\item
  Bail, Christopher, Taylor Brown, and Andreas Wimmer. 2019. ``Prestige,
  Proximity, and Prejudice: How Google Search Terms Diffuse across the
  World.'' \emph{American Journal of Sociology} 124 (5): 1496--1548.
  \url{https://doi.org/10.1086/702007}.
\item
  Bail, Christopher, Taylor Brown, and Marcus Mann. 2017. ``Channeling
  Hearts and Minds: Advocacy Organizations, Cognitive-Emotional
  Currents, and Public Conversation.'' \emph{American Sociological
  Review} 82(6): 1188-1213.
\item
  Mitts, Tamar. 2019. ``From Isolation to Radicalization: Anti-Muslim
  Hostility and Support for ISIS in the West.'' \emph{American Political
  Science Review} 113 (1): 173--94.
  \url{https://doi.org/10.1017/S0003055418000618}.
\item
  Munger, Kevin, and Joseph Phillips. 2020. ``Right-Wing YouTube: A
  Supply and Demand Perspective.'' \emph{The International Journal of
  Press/Politics}, 34.
\item
  Gross, Neil and Mann, Marcus.2017. ``Is There a Ferguson Effect?
  Google Searches, Concern About Police Violence, and Crime in U.S.
  Cities, 2014-2016.'' Socius. 3, 1-16.
\end{itemize}

\hypertarget{week-4---for-918}{%
\subsection{Week 4 - For 9/18}\label{week-4---for-918}}

\hypertarget{data-collection-ii-online-experiments-and-surveys}{%
\subsubsection{Data Collection II: Online experiments and
surveys}\label{data-collection-ii-online-experiments-and-surveys}}

\emph{Readings}

\begin{itemize}
\tightlist
\item
  \emph{Bit by Bit}, C3-5
\item
  Salganik, Matthew J., and Duncan J. Watts. 2008. ``Leading the Herd
  Astray: An Experimental Study of Self-Fulfilling Prophecies in an
  Artificial Cultural Market.'' \emph{Social Psychology Quarterly} 71
  (4): 338--55. \url{https://doi.org/10.1177/019027250807100404}.
\item
  Kramer, Adam D. I., Jamie E. Guillory, and Jeffrey T. Hancock. 2014.
  ``Experimental Evidence of Massive-Scale Emotional Contagion through
  Social Networks.'' \emph{Proceedings of the National Academy of
  Sciences 111} (24): 8788--90.
  \url{https://doi.org/10.1073/pnas.1320040111}.
\item
  Munger, Kevin. 2016. ``Tweetment Effects on the Tweeted:
  Experimentally Reducing Racist Harassment.'' \emph{Political
  Behavior}, November. \url{https://doi.org/10.1007/s11109-016-9373-5}.
\item
  Wang, Wei, David Rothschild, Sharad Goel, and Andrew Gelman. 2015.
  ``Forecasting Elections with Non-Representative Polls.''
  \emph{International Journal of Forecasting} 31 (3): 980--91.
  \url{https://doi.org/10.1016/j.ijforecast.2014.06.001}.
\item
  Christopher Bail, Lisa Argyle, Taylor W. Brown, John Bumpus, Haohan
  Chen, M.B. Fallin Hunzaker, Jaemin Lee, Marcus Mann, Friedolin
  Merhout, and Alexander Volfovsky. 2018. ``Exposure to Opposing Views
  can Increase Political Polarization: Evidence from a Large-Scale Field
  Experiment on Social Media.'' Proceedings of the National Academy of
  Sciences.
\end{itemize}

\hypertarget{week-5---for-925}{%
\subsection{Week 5 - For 9/25}\label{week-5---for-925}}

\hypertarget{natural-language-processing-i-fundamentals}{%
\subsubsection{Natural Language Processing I:
Fundamentals}\label{natural-language-processing-i-fundamentals}}

\emph{Readings}

\begin{itemize}
\tightlist
\item
  \emph{Text Mining with R}, C1, 3-5
\item
  Grimmer, Justin, and Brandon Stewart. 2013. ``Text as Data: The
  Promise and Pitfalls of Automatic Content Analysis Methods for
  Political Texts.'' \emph{Political Analysis} 21 (3): 267--97.
  \url{https://doi.org/10.1093/pan/mps028}.
\item
  DiMaggio, Paul. 2015. ``Adapting Computational Text Analysis to Social
  Science (and Vice Versa).'' \emph{Big Data \& Society} 2 (2):
  205395171560290. \url{https://doi.org/10.1177/2053951715602908}.
\item
  Evans, James, and Pedro Aceves. 2016. ``Machine Translation: Mining
  Text for Social Theory.'' \emph{Annual Review of Sociology} 42 (1):
  21--50. \url{https://doi.org/10.1146/annurev-soc-081715-074206}.
\item
  Austin Van Loon. 2022. ``Three Families of Automated Text Analysis''.
  Working Paper. \url{https://osf.io/preprints/socarxiv/htnej/}
\end{itemize}

\emph{Recommended}

\begin{itemize}
\tightlist
\item
  \emph{Speech and Language Processing}, C6, pages 1-13.
\item
  Danescu-Niculescu-Mizil, Cristian, Lillian Lee, Bo Pang, and Jon
  Kleinberg. 2012. ``Echoes of Power: Language Effects and Power
  Differences in Social Interaction.'' In \emph{Proceedings of the 21st
  International Conference on World Wide Web}, 699--708. ACM.
  \url{http://dl.acm.org/citation.cfm?id=2187931}.
\item
  Danescu-Niculescu-Mizil, Cristian, Robert West, Dan Jurafsky, Jure
  Leskovec, and Christopher Potts. 2013. ``No Country for Old Members:
  User Lifecycle and Linguistic Change in Online Communities.'' In
  \emph{Proceedings of the 22nd International Conference on World Wide
  Web}, 307--318. \url{http://dl.acm.org/citation.cfm?id=2488416}.
\item
  Niculae, Vlad, Srijan Kumar, Jordan Boyd-Graber, and Cristian
  Danescu-Niculescu-Mizil. 2015. ``Linguistic Harbingers of Betrayal: A
  Case Study on an Online Strategy Game.'' In \emph{Proceedings of the
  53rd Annual Meeting of the Association for Computational Linguistics
  and the 7th International Joint Conference on Natural Language
  Processing}. Beijing, China: ACL.
\end{itemize}

\hypertarget{week-6---for-102}{%
\subsection{Week 6 - For 10/2}\label{week-6---for-102}}

\hypertarget{natural-language-processing-ii-dictionaries-word-embeddings-topic-models-and-networks}{%
\subsubsection{Natural Language Processing II: Dictionaries, word
embeddings, topic models, and
networks}\label{natural-language-processing-ii-dictionaries-word-embeddings-topic-models-and-networks}}

\emph{Readings}

\begin{itemize}
\tightlist
\item
  \emph{Text Mining with R}: C5.
\item
  \emph{Speech and Language Processing}, C6, pages 17-30.
\item
  Mikolov, Tomas, Ilya Sutskever, Kai Chen, Greg Corrado, and Jeff Dean.
  2013. ``Distributed Representations of Words and Phrases and Their
  Compositionality.'' In \emph{Advances in Neural Information Processing
  Systems}, 3111--3119.
  \url{http://papers.nips.cc/paper/5021-distributed-representations}.
\item
  Kozlowski, Austin, Matt Taddy, and James Evans. 2019. ``The Geometry
  of Culture: Analyzing the Meanings of Class through Word Embeddings.''
  \emph{American Sociological Review}, September, 000312241987713.
  \url{https://doi.org/10.1177/0003122419877135}.
\item
  Arseniev-Koehler, Alina, and Jacob G. Foster. 2020. ``Machine Learning
  as a Model for Cultural Learning: Teaching an Algorithm What It Means
  to Be Fat.'' Working Paper. \emph{SocArXiv}.
  \url{https://doi.org/10.31235/osf.io/c9yj3}.
\item
  \emph{Text Mining with R}: C6.
\item
  Mohr, John, and Petko Bogdanov. 2013. ``Introduction---Topic Models:
  What They Are and Why They Matter.'' \emph{Poetics} 41 (6): 545--69.
  \url{https://doi.org/10.1016/j.poetic.2013.10.001}.
\item
  DiMaggio, Paul, Manish Nag, and David Blei. 2013. ``Exploiting
  Affinities between Topic Modeling and the Sociological Perspective on
  Culture: Application to Newspaper Coverage of U.S. Government Arts
  Funding.'' \emph{Poetics} 41 (6): 570--606.
  \url{https://doi.org/10.1016/j.poetic.2013.08.004}.
\item
  Roberts, Margaret, Brandon M. Stewart, Dustin Tingley, Christopher
  Lucas, Jetson Leder-Luis, Shana Kushner Gadarian, Bethany Albertson,
  and David Rand. 2014. ``Structural Topic Models for Open-Ended Survey
  Responses: Structural Topic Models for Survey Responses.''
  \emph{American Journal of Political Science} 58 (4): 1064--82.
  \url{https://doi.org/10.1111/ajps.12103}.
\item
  Karell, Daniel, and Michael Freedman. 2019. ``Rhetorics of
  Radicalism.'' \emph{American Sociological Review} 84 (4): 726--53.
  \url{https://doi.org/10.1177/0003122419859519}.
\item
  Blei, David 2012. ``Probabilistic Topic Models.'' \emph{Communications
  of the ACM} 55 (4): 77. \url{https://doi.org/10.1145/2133806.2133826}.
\item
  Stoltz, Dustin S, and Marshall A Taylor. 2019. ``Textual Spanning:
  Finding Discursive Holes in Text Networks.'' \emph{Socius:
  Sociological Research for a Dynamic World} .
\item
  Hamilton, William, Jure Leskovec, and Dan Jurafsky. 2016. ``Diachronic
  Word Embeddings Reveal Statistical Laws of Semantic Change.'' In
  \emph{Proceedings of the 54th Annual Meeting of the Association for
  Computational Linguistics}, 1489--1501.
\item
  Devlin, Jacob, Ming-Wei Chang, Kenton Lee, and Kristina Toutanova.
  2019. ``BERT: Pre-Training of Deep Bidirectional Transformers for
  Language Understanding.'' In \emph{Proceedings of NAACL-HLT 2019},
  4171--86. ACL.
\item
  Manning, Christopher D., Kevin Clark, John Hewitt, Urvashi Khandelwal,
  and Omer Levy. 2020. ``Emergent Linguistic Structure in Artificial
  Neural Networks Trained by Self-Supervision.'' \emph{Proceedings of
  the National Academy of Sciences}, June, 201907367.
  \url{https://doi.org/10.1073/pnas.1907367117}.
\item
  Rodman, Emma. 2019. ``A Timely Intervention: Tracking the Changing
  Meanings of Political Concepts with Word Vectors.'' \emph{Political
  Analysis}l, July, 1--25. \url{https://doi.org/10.1017/pan.2019.23}.
\end{itemize}

\hypertarget{week-7---for-109}{%
\subsection{Week 7 - For 10/9}\label{week-7---for-109}}

\hypertarget{data-visualization}{%
\subsubsection{Data Visualization}\label{data-visualization}}

\emph{Readings}

\begin{itemize}
\tightlist
\item
  Healy, Kieran. 2018. \emph{Data Visualization: A Practical
  Introduction}. Princeton University Press. \url{https://socviz.co/}
\end{itemize}

\hypertarget{final-paper}{%
\section{Final paper}\label{final-paper}}

\emph{Final paper due on Saturday 10/20 at 5pm ET}

\end{document}
